\href{https://travis-ci.org/SoPra-Team-10/Server}{\tt !\mbox{[}Build Status\mbox{]}(https\-://travis-\/ci.\-org/\-So\-Pra-\/\-Team-\/10/\-Server.\-svg?branch=master)} \section*{Server}

Server component for the Quidditch Game.

\subsection*{Getting started}

You can choose between using Docker or manually installing all dependencies. Docker is the preferred method as it already installs the toolchain and all dependencies.

\subsubsection*{Docker}

In the root directory of the project build the docker image (\char`\"{}server\char`\"{} is the name of the container, this can be replaced by a different name)\-: ``` docker build -\/t server . ```

Now start the container, you need to map the internal port (8080 by default, to some external port 80 in this case) and map the external file (match.\-json) to an internal file\-: ``` docker run -\/v /match.json\-:match.\-json -\/p 80\-:8080 server ./\-Server -\/m /match.json -\/p 8080 ``` That's it you should now have a running docker instance.

\subsubsection*{Manually installing the Server}

If you need to debug the server it can be easier to do this outside of docker.

\subsubsection*{Prerequisites}


\begin{DoxyItemize}
\item A C++17 compatible Compiler (e.\-g. G\-C\-C-\/8)
\item C\-Make (min 3.\-10) and G\-N\-U-\/\-Make
\item Adress-\/\-Sanitizer for run time checks
\item \href{https://github.com/SoPra-Team-10/Network}{\tt Sopra\-Network}
\item \href{https://github.com/SoPra-Team-10/GameLogic}{\tt Sopra\-Game\-Logic}
\item \href{https://github.com/SoPra-Team-10/Messages}{\tt Sopra\-Messages}
\item Either a P\-O\-S\-I\-X-\/\-Compliant O\-S or Cygwin (to use pthreads)
\item Optional\-: Google Tests and Google Mock for Unit-\/\-Tests
\end{DoxyItemize}

\subsubsection*{Compiling the Application}

In the root directory of the project create a new directory (in this example it will be called build), change in this directory.

Next generate a makefile using cmake\-: ``` cmake .. ``` if any error occurs recheck the prerequisites. Next compile the program\-: ``` make ``{\ttfamily  you can now run the server by executing the created}Server{\ttfamily file\-: }`` ./\-Server ```

\subsection*{Log-\/\-Levels}

\begin{TabularC}{3}
\hline
\rowcolor{lightgray}{\bf Log-\/\-Level }&{\bf Color }&{\bf Explanation  }\\\cline{1-3}
0 &-\/ &No log messages \\\cline{1-3}
1 &Red &Only error messages \\\cline{1-3}
2 &Yellow &Error messages and warning \\\cline{1-3}
3 &Blue &Error messages, warning and info messages \\\cline{1-3}
4 &White &All messages (error, warning, info and debug) \\\cline{1-3}
\end{TabularC}
\subsection*{External Librarys}


\begin{DoxyItemize}
\item \href{https://github.com/SoPra-Team-10/Network}{\tt Sopra\-Network}
\item \href{https://github.com/SoPra-Team-10/GameLogic}{\tt Sopra\-Game\-Logic}
\item \href{https://github.com/nlohmann/json}{\tt nlohmann\-::json}
\end{DoxyItemize}

\subsection*{Doxygen Dokumentation}


\begin{DoxyItemize}
\item \href{https://sopra-team-10.github.io/Server/master/html/index.html}{\tt Master Branch}
\item \href{https://sopra-team-10.github.io/Server/develop/html/index.html}{\tt Develop Branch} 
\end{DoxyItemize}